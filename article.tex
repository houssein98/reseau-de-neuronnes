\documentclass[a4paper,10pt]{article}
\usepackage[utf8]{inputenc}
\usepackage[T1]{fontenc}
\usepackage[frenchb]{babel}
\usepackage{numprint}

%opening
\title{Apprentissage par des réseaux de neurones}
\author{E. Beghdadi, M. Marrakchi Benazzouz, T. Chabal, Y. Vanlaer, T. Giraudon}

\begin{document}

\maketitle

\begin{abstract}

\end{abstract}

\section{Introduction}
En 1963, l'Américain Donald Michie créé Menace, Machine Educable Noughts And Crosses Engine, une machine capable de rivaliser avec des joueurs humains
au jeu du Tic-Tac-Toe, plus connu en France sous le nom de morpion.
Pour réaliser cette prouesse technologique, il utilise alors la technique de ``l'apprentissage par renforcement``. Celle-ci consiste à faire jouer la machine un grand
nombre de fois contre un joueur réel et à apprendre de ces parties en corrigeant sa stratégie au fur et à mesure. Cette réussite signe alors la naissance du machine
learning tel que nous le connaissons aujourd'hui. \\
Dans les années 1970-1980, beaucoup de progrès théoriques sont réalisés : on imagine ainsi un système directement inspiré du vivant, le réseau de neurones.
\section{Résultats}

\section{Discussion}

\section{Conclusion}


\end{document}
